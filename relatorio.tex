%% abtex2-modelo-artigo.tex, v-1.9.5 laurocesar
%% Copyright 2012-2015 by abnTeX2 group at http://www.abntex.net.br/ 
%%
%% This work may be distributed and/or modified under the
%% conditions of the LaTeX Project Public License, either version 1.3
%% of this license or (at your option) any later version.
%% The latest version of this license is in
%%   http://www.latex-project.org/lppl.txt
%% and version 1.3 or later is part of all distributions of LaTeX
%% version 2005/12/01 or later.
%%
%% This work has the LPPL maintenance status `maintained'.
%% 
%% The Current Maintainer of this work is the abnTeX2 team, led
%% by Lauro César Araujo. Further information are available on 
%% http://www.abntex.net.br/
%%
%% This work consists of the files abntex2-modelo-artigo.tex and
%% abntex2-modelo-references.bib
%%

% ------------------------------------------------------------------------
% ------------------------------------------------------------------------
% abnTeX2: Modelo de Artigo Acadêmico em conformidade com
% ABNT NBR 6022:2003: Informação e documentação - Artigo em publicação 
% periódica científica impressa - Apresentação
% ------------------------------------------------------------------------
% ------------------------------------------------------------------------

\documentclass[
	% -- opções da classe memoir --
	article,			% indica que é um artigo acadêmico
	11pt,				% tamanho da fonte
	oneside,			% para impressão apenas no verso. Oposto a twoside
	a4paper,			% tamanho do papel. 
	% -- opções da classe abntex2 --
	%chapter=TITLE,		% títulos de capítulos convertidos em letras maiúsculas
	%section=TITLE,		% títulos de seções convertidos em letras maiúsculas
	%subsection=TITLE,	% títulos de subseções convertidos em letras maiúsculas
	%subsubsection=TITLE % títulos de subsubseções convertidos em letras maiúsculas
	% -- opções do pacote babel --
	english,			% idioma adicional para hifenização
	brazil,				% o último idioma é o principal do documento
	sumario=tradicional,
	fleqn				%Equações no lado esquerdo
	]{abntex2}


% ---
% PACOTES
% ---

% ---
% Pacotes fundamentais 
% ---
\usepackage{lmodern}			% Usa a fonte Latin Modern
\usepackage[T1]{fontenc}		% Selecao de codigos de fonte.
\usepackage[utf8]{inputenc}		% Codificacao do documento (conversão automática dos acentos)
\usepackage{indentfirst}		% Indenta o primeiro parágrafo de cada seção.
\usepackage{nomencl} 			% Lista de simbolos
\usepackage{color}				% Controle das cores
\usepackage{graphicx}			% Inclusão de gráficos
\usepackage{microtype} 			% para melhorias de justificação
% ---
		
% ---
% Pacotes de citações
% ---
\usepackage[alf]{abntex2cite}	% Citações padrão ABNT
% ---

\usepackage{amsmath}

\newcommand{\coment}{\textbackslash\textbackslash}

% ---
% Informações de dados para CAPA e FOLHA DE ROSTO
% ---
\titulo{Trabalho KNN \\ Relatório}
\autor{Leandro Duarte Pulgatti}
\local{Brasil}
\data{\the\year}
% ---

% ---
% Configurações de aparência do PDF final

% alterando o aspecto da cor azul
\definecolor{blue}{RGB}{41,5,195}

% informações do PDF
\makeatletter
\hypersetup{
     	%pagebackref=true,
		pdftitle={\@title}, 
		pdfauthor={\@author},
    	pdfsubject={Modelo de artigo científico com abnTeX2},
	    pdfcreator={LaTeX with abnTeX2},
		pdfkeywords={abnt}{latex}{abntex}{abntex2}{atigo científico}, 
		colorlinks=true,       		% false: boxed links; true: colored links
    	linkcolor=blue,          	% color of internal links
    	citecolor=blue,        		% color of links to bibliography
    	filecolor=magenta,      		% color of file links
		urlcolor=blue,
		bookmarksdepth=4
}
\makeatother
% --- 

% ---
% compila o indice
% ---
\makeindex
% ---

% ---
% Altera as margens padrões
% ---
\setlrmarginsandblock{3cm}{3cm}{*}
\setulmarginsandblock{3cm}{3cm}{*}
\checkandfixthelayout
% ---

% --- 
% Espaçamentos entre linhas e parágrafos 
% --- 

\setlength\parindent{0pt} %sem identação para a primeira linha do parágrafo

% Controle do espaçamento entre um parágrafo e outro:
\setlength{\parskip}{0.2cm}  % tente também \onelineskip

% Espaçamento simples
\SingleSpacing

% ----
% Início do documento
% ----
\begin{document}

% Seleciona o idioma do documento (conforme pacotes do babel)
%\selectlanguage{english}
\selectlanguage{brazil}

% Retira espaço extra obsoleto entre as frases.
\frenchspacing 

\maketitle


% ---

% ----------------------------------------------------------
% ELEMENTOS TEXTUAIS
% ----------------------------------------------------------
\textual

\section{Descrição}

Relatório de experimento apresentado na disciplina de Aprendizado de Máquina - INFO7004.


Este documento tem por finalidade descrever as ações realizadas para realizar o trabalho proposto e listar as dificuldades encontradas bem como as soluções aplicadas.


A primeira parte foi a extração de características dos textos da base IMDB.

A base original possuía 100.000 registros, sendo:

\begin{itemize}
	\item 25.000 rotulados como "neg"
	\item 25.000 rotulados como "pos"
	\item 50.000 sem rótulo "unsup'
\end{itemize}

Os 50.000 registros não rotulados foram descartados.
Devido a maneira como o programa KNN foi implementado foi necessário alterar os label's para valores numéricos.
Assim os dados rotulados como "neg" foram alterados para 0 (zero) e os rotulaos como "pos"  para 1 (um).

Para este fim foi criado um programa na linguagem python utilizando-se da biblioteca gensim[1] que implementa o modelo Word2vec.

Esta biblioteca cria um vetor de vetores, onde cada vetor interno representa uma palavra contida no texto.
Foi realizada a soma dos vetores para se chegar a um vetor que representasse as frases como um todo.

Foram utilizadas duas diferentes configurações para a função que gera os vetores na tentativa de se conseguir um vetor de características mais representativo.

\begin{itemize}
	\item size=100, window=5, min\_count=1  -- indica um vetor com 100 características e uma janela de 5 palavras

    \item size=150, window=10, min\_count=2  -- indica um vetor com 150 características e uma janela de 10 palavras, apenas das palavras que aparecem pelo menos duas vezes no texto \footnote{Devido a erros com a configuração min\_count=2, pois algumas linhas não continham pelo menos uma palavra repetida duas vezes, foram retiradas 3400 linhas que possuíam tamanho de frase inferior a 400 caracteres}
\end{itemize}

Os dados foram então misturados (shuffle) pois os mesmos se apresentavam sequenciais em relação ao label, apesar dos classificadores baseados em distância não sofrem com problemas este passo facilitou a separação das bases.

Os arquivos com as características geradas foram então separados em duas bases de igual tamanho para trenamento e testes.

Foi adaptado um programa criado em C++ para a disciplina CI762 cursada no segundo semestre de 2014.

\section{Resultados}

A execução do programa contra as bases revelou acertos próximos a 50\% ( 53,54\% no melhor caso ).
Mesmo a alteração do valor de K {1,3,5,10} não resultou em melhoras na taxa de predição.
Como os resultados foram muito próximos ao esperado de um sorteio, foi realizada uma validação com os mesmos dados na ferramenta Weka[2] sendo obtidos valores muito próximos a 50\%.


\section{Considerações}

Devido a baixíssima taxa de acerto verificou-se que as características extraídas claramente não são suficientemente discriminantes.
Assim a implementação de outros cálculos de distância (Manhatan e Distância Cosseno) acabou não sendo testada.

Algumas ações podem ser realizadas para uma melhor taxa de acerto.

\begin{itemize}
	\item Limpeza dos dados ( separation words, case ) 
	\item normalização dos vetores gerados \footnote{ Utilizar a distância de Mahalanobis dispensa este passo pois a mesma é invariante escala das medições.}
	\item Utilizar outros tamanhos de vetores, pois vetotres com 100 e 150 posições apresentaram pouca variação do resultado.
	\item Utilizar janelas maiores.
\end{itemize}


\postextual

% ----------------------------------------------------------
% Referências bibliográficas
% ----------------------------------------------------------
\bibliography{referencias}

[1] https://radimrehurek.com/gensim/models/word2vec.html

[2] https://www.cs.waikato.ac.nz/ml/weka/

\end{document}
